\documentclass[11pt]{article}
\usepackage[letterpaper, margin=1in]{geometry}

\usepackage[space]{cite}
\usepackage{amsmath}
\usepackage{amsfonts}
\usepackage{amssymb}
\usepackage{mathrsfs}
\usepackage{xspace}
\usepackage{xparse}

\let\labelindent\relax
\usepackage{enumitem}

\usepackage{algorithm}
\usepackage{algpseudocode}

\usepackage{tabularx}
\usepackage{booktabs}
\usepackage{multirow}

\usepackage{tikz}
\usetikzlibrary{positioning}

\usepackage{epstopdf}
\epstopdfsetup{update}

\usepackage{listings}
\usepackage{xcolor}

\usepackage[colorlinks]{hyperref}
\usepackage[symbols,nogroupskip,sort=use]{glossaries-extra}

%%%%%%%%%%%%%%%%%%%%%%%%%%%%%%%%%%%%%%%%%%%%%%%%%%%%%%%%%%%%%%%%%%%%%%%%%%%%%%%%

\definecolor{codegreen}{rgb}{0,0.6,0}
\definecolor{codegray}{rgb}{0.5,0.5,0.5}
\definecolor{codepurple}{rgb}{0.58,0,0.82}

\lstdefinestyle{mystyle}{
    % backgroundcolor=\color{backcolour},
    commentstyle=\color{codegreen},
    % keywordstyle=\color{magenta},
    numberstyle=\tiny\color{codegray},
    stringstyle=\color{codepurple},
    basicstyle=\ttfamily\footnotesize,
    breakatwhitespace=false,
    breaklines=true,
    captionpos=b,
    keepspaces=true,
    numbers=left,
    numbersep=5pt,
    showspaces=false,
    showstringspaces=false,
    showtabs=false,
    tabsize=4
}

\lstset{style=mystyle}

%%%%%%%%%%%%%%%%%%%%%%%%%%%%%%%%%%%%%%%%%%%%%%%%%%%%%%%%%%%%%%%%%%%%%%%%%%%%%%%%

\makenoidxglossaries

\glsxtrnewsymbol[description={Sample Set}]{S}{%
  \ensuremath{\mathcal{S}}
}

\glsxtrnewsymbol[description={Markov Control Process}]{H}{%
  \ensuremath{\mathcal{H}}
}
\glsxtrnewsymbol[description={State Space}]{X}{%
  \ensuremath{\mathcal{X} \subseteq \Re^{n}}
}
\glsxtrnewsymbol[description={Control Space}]{U}{%
  \ensuremath{\mathcal{U} \subseteq \Re^{m}}
}
\glsxtrnewsymbol[description={Stochastic Kernel}]{Q}{%
  \ensuremath{Q}
}

%%%%%%%%%%%%%%%%%%%%%%%%%%%%%%%%%%%%%%%%%%%%%%%%%%%%%%%%%%%%%%%%%%%%%%%%%%%%%%%%

\title{%
  Documentation for
  ``Stochastic Reachability for Systems up to a Million Dimensions''
}
\author{Adam J. Thorpe, Vignesh Sivaramakrishnan, Meeko M. K. Oishi}


\begin{document}

%%%%%%%%%%%%%%%%%%%%%%%%%%%%%%%%%%%%%%%%%%%%%%%%%%%%%%%%%%%%%%%%%%%%%%%%%%%%%%%%

\maketitle

%%%%%%%%%%%%%%%%%%%%%%%%%%%%%%%%%%%%%%%%%%%%%%%%%%%%%%%%%%%%%%%%%%%%%%%%%%%%%%%%

Documentation for the algorithms presented in ``Stochastic Reachability for Systems up to a Million Dimensions'' by Adam J. Thorpe, Vignesh Sivaramakrishnan, Meeko M. K. Oishi.

\tableofcontents

\printnoidxglossary[type=symbols,style=long,title={List of Symbols}]

\newpage

%%%%%%%%%%%%%%%%%%%%%%%%%%%%%%%%%%%%%%%%%%%%%%%%%%%%%%%%%%%%%%%%%%%%%%%%%%%%%%%%

\section{Start Here}

%%%%%%%%%%%%%%%%%%%%%%%%%%%%%%%%%%%%%%%%%%%%%%%%%%%%%%%%%%%%%%%%%%%%%%%%%%%%%%%%

\section{Instructions}

\subsection{Running The Code}

\subsection{Generating the Figures}

\subsection{Modifying the Code}

%%%%%%%%%%%%%%%%%%%%%%%%%%%%%%%%%%%%%%%%%%%%%%%%%%%%%%%%%%%%%%%%%%%%%%%%%%%%%%%%

\section{Algorithms}

The algorithms presented

%%%%%%%%%%%%%%%%%%%%%%%%%%%%%%%%%%%%%%%%%%%%%%%%%%%%%%%%%%%%%%%%%%%%%%%%%%%%%%%%

\subsection{Preliminaries}

We consider a a Markov control process $\gls{H}$, which is defined in \cite{summers} as a 3-tuple:
\begin{align}
  \mathcal{H} = (\mathcal{X}, \mathcal{U}, Q)
\end{align}
where $\gls{X}$ is the state space, $\gls{U}$ is the control space, and $\gls{Q}$ is a stochastic kernel
$Q : \mathscr{B}(\mathcal{X}) \times \mathcal{X} \times \mathcal{U} \rightarrow [0, 1]$, which is a Borel-measurable function that maps a probability measure $Q(\cdot \,|\, x, u)$ to each $x \in \mathcal{X}$ and $u \in \mathcal{U}$ in the Borel space $(\mathcal{X}, \mathscr{B}(\mathcal{X}))$.
A Markov control process can describe a wide class of stochastic, time-invariant systems, that can have either linear or nonlinear dynamics, as well as non-Gaussian disturbances.
%
We consider a set $\gls{S}$ of $M$ samples of the form
$\lbrace (\bar{x}_{i}, \bar{u}_{i}, \bar{y}_{i}) \rbrace_{i=1}^{M}$ taken from the stochastic kernel,
such that $\bar{y}_{i}$ is drawn i.i.d. from the stochastic kernel $Q$,
and $\bar{u}_{i}$ is drawn from a fixed Markov policy $\pi$.
\begin{align}
	\bar{y}_{i} &\sim Q(\,\cdot\,|\,\bar{x}_{i}, \bar{u}_{i}) \\
	\bar{u}_{i} &= \pi(\bar{x}_{i})
\end{align}
The samples can be generated experimentally or via simulation, meaning they can be taken from real observations of the system evolution, or they can be generated using a known model. For demonstration purposes, all examples use samples collected via simulation.
%
Once the samples are generated, the algorithm assumes no knowledge of the stochastic kernel $Q$ or the disturbance.

%%%%%%%%%%%%%%%%%%%%%%%%%%%%%%%%%%%%%%%%%%%%%%%%%%%%%%%%%%%%%%%%%%%%%%%%%%%%%%%%

\subsection{Kernel Distribution Embeddings Backward Recursion Algorithm}


%%%%%%%%%%%%%%%%%%%%%%%%%%%%%%%%%%%%%%%%%%%%%%%%%%%%%%%%%%%%%%%%%%%%%%%%%%%%%%%%

\subsection{Kernel Distribution Embeddings Backward Recursion (RFF) Algorithm}


%%%%%%%%%%%%%%%%%%%%%%%%%%%%%%%%%%%%%%%%%%%%%%%%%%%%%%%%%%%%%%%%%%%%%%%%%%%%%%%%

\section{Problems}


%%%%%%%%%%%%%%%%%%%%%%%%%%%%%%%%%%%%%%%%%%%%%%%%%%%%%%%%%%%%%%%%%%%%%%%%%%%%%%%%

\subsection{Terminal-Hitting Time Problem}


%%%%%%%%%%%%%%%%%%%%%%%%%%%%%%%%%%%%%%%%%%%%%%%%%%%%%%%%%%%%%%%%%%%%%%%%%%%%%%%%

\subsection{First-Hitting Time Problem}


%%%%%%%%%%%%%%%%%%%%%%%%%%%%%%%%%%%%%%%%%%%%%%%%%%%%%%%%%%%%%%%%%%%%%%%%%%%%%%%%

\section{Systems}

The algorithms accept sample data drawn from a stochastic kernel $\gls{Q}$. The data should be formatted such that the realizations of the stochastic kernel are formatted into the columns of a sample vector
\begin{align}
  \bar{x} = [\bar{x}_{1}, \ldots, \bar{x}_{M}] \\
  \bar{u} = [\bar{u}_{1}, \ldots, \bar{u}_{M}] \\
  \bar{y} = [\bar{y}_{1}, \ldots, \bar{y}_{M}]
\end{align}
where the number of columns is $M$, and the number of rows is the dimensionality of the samples. For example, if $\bar{x}_{i}, \bar{y}_{i} \in \Re^{n}$, $\bar{x}$ and $\bar{y}$ should be $[n \times M]$.

%%%%%%%%%%%%%%%%%%%%%%%%%%%%%%%%%%%%%%%%%%%%%%%%%%%%%%%%%%%%%%%%%%%%%%%%%%%%%%%%

\subsection{System Samples}

The system input to the algorithms is a set of samples organized in a class called \verb|SystemSamples|.
For example, to generate samples for the discrete-time double integrator system
with sampling time $T = 0.25$,
\begin{align}
  \boldsymbol{x}_{k+1} =
  \begin{bmatrix}
    1 & T \\
    0 & 1
  \end{bmatrix}
  \boldsymbol{x}_{k} +
  \begin{bmatrix}
    \frac{T^{2}}{2!} \\
    T
  \end{bmatrix}
  u_{k} +
  \boldsymbol{w}_{k}
\end{align}

\begin{lstlisting}[language=Matlab]
  % Dimensionality of the state space samples.
  n = 2;
  % Dimensionality of the input space samples.
  m = 2;
  % Sampling time.
  T = 0.25;

  % Number of samples.
  M = 1000;

  % Compute random initial states sampled from a zero-mean Gaussian.
  X = randn(n, M);
  % Compute the input samples. For this example, the input is chosen to be 0.
  U = zeros(m, M);
  % Compute the disturbance.
  W = randn(n, M);

  % Construct the state and input matrices.
  A = [1 T; 0 1];
  B = [(T^2)/2!; T];

  % compute the output samples.
  Y = A*X + B*U + W;

  % Create a SystemSamples object.
  samples = SystemSamples('X', X, 'U', U, 'Y', Y);
\end{lstlisting}

%%%%%%%%%%%%%%%%%%%%%%%%%%%%%%%%%%%%%%%%%%%%%%%%%%%%%%%%%%%%%%%%%%%%%%%%%%%%%%%%

\bibliographystyle{plain}
\bibliography{bibliography}

%%%%%%%%%%%%%%%%%%%%%%%%%%%%%%%%%%%%%%%%%%%%%%%%%%%%%%%%%%%%%%%%%%%%%%%%%%%%%%%%

\end{document}
